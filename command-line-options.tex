%&pdflatex
\documentclass{article}
\usepackage[british]{babel}
\usepackage[
    includeall,
    marginpar=1.5in,
    hscale=0.707,vscale=0.9
    ]{geometry}
\usepackage{hyperref}
\title{Unix Shell Scripting \\ Command Line Options}
\author{%
Dr Alun Moon\thanks{\url{alun.moon@northumbria.ac.uk}}\\
Dr David Kendall\thanks{\url{david.kendall@northumbria.ac.uk}}\\
Dr Michael Brockway\thanks{\url{michael.brockway@northumbria.ac.uk}}\\
Dr Neil Elliot\thanks{\url{neil.elliot@northumbria.ac.uk}}
}

\usepackage{fancyvrb,listings,verbatim,tcolorbox}

\newcommand{\required}[1]{$\langle$\itshape #1$\rangle$}
\newcommand{\optional}[1]{$[$\itshape #1$]$}
\begin{document}
\maketitle
\newpage
\section{Command line options and scripts}
Shell scripts are almost always \emph{not-interactive}.
User input to the script is provided using command-line options and parameters.

\url{http://pubs.opengroup.org/onlinepubs/9699919799/basedefs/V1_chap12.html}

\paragraph{Examples} are
\begin{verbatim}
ls -l
head -n5
grep -v
\end{verbatim}

\subsection{Conventions}
The Unix convention is that options are command-line parameters that start with the hyphen/minus-sign (\textsc{ascii} 45).  There are three common conventions for the form of an option

\paragraph{Unix} options are a single letter, options that require a parameter (such as a file name) have the option letter as the next command-line parameter.
\subparagraph{examples}
\begin{verbatim}
  ls -l
  head -n10
  head -n 5
\end{verbatim}

\paragraph{GNU} style options staet with two hyphens and are a descriptive word (if more than one word hyphens separate words).  Parameters for the option can be \emph{either} the next command-line parameter, or be part of the option with an equals sign (\textsc{ascii} 61) joining the parts.
\subparagraph{examples}
\begin{verbatim}
  tar --extract --file=archive.tar
\end{verbatim}

\paragraph{X} The X11 toolkit introduced another convention, options are a single hyphen followed by a word, any parameters are the next command-line parameter
\subparagraph{examples}
\begin{verbatim}
  xterm -geometry 100x25
  java -jar Lander.jar
\end{verbatim}

\section{POSIX style options}
The POSIX standard guidelines for untilities say that options should follow the Unix notation above.
\section{Processing script parameters (positional parameters)}
Fortunately shells provide some tools that assist with the processing of options, the built-in command \verb'getopts' and the command \verb'getopt'  (usually \path{/usr/bin/getopt})

\subsection{Simple options with \texttt{getopts}}
The built-in function \texttt{getopts} can help handle simple single character options.  These options and any parameters \emph{must} be the first parameters on the commandline.

\begin{itemize}
  \item Options start with a hyphen `\texttt{-}'
  \item Options are identified by a single character \\
        (except colon `\texttt{:}' or question-mark '\texttt{?}`)
  \item Where options take a parameter, the value \emph{must} follow the option character and may be seperated by whitespace.
  \item Multiple options can be combined into a single string following the hyphen
\end{itemize}

The \texttt{getopts} command has the form
\begin{quote}
\texttt{getopts \required{opt-string} \required{var-name}}
\end{quote}
Where
\begin{description}
  \item[opt-string] is a string containing the option characters to recognise, where an option expects a parameter it is followed by a colon `\texttt{:}'
  \item[var-name] is the name of a shell variable to set with the found option
\end{description}
\texttt{getopts} returns \textit{true} if it has successfully found an option, and \textit{false} when option proccessing is finished.
\paragraph{Variables} that are set by \texttt{getopts} are
\begin{description}
  \item[\ttfamily{OPTARG}] is set to the value of the option parameter that has been found
\item[\ttfamily{OPTIND}] is the position of the next parameter to process
\end{description}
When option processing has finished \texttt{OPTIND} is set to the position of the first non-option parameter.

\subsubsection{Example}
\begin{Verbatim}[numbers=left,gobble=2,commandchars=\\\{\}]
  #!/usr/bin/bash
  while getopts hf:m option   \label{getopt:use}
  do
    case $option in
      h)
        echo "Help text...."
        ;;
      m)
        flag=
        ;;
      f)
        filename=$OPTARG
        ;;
    esac
  done
  shift $((OPTIND-1))
  for param in $*
  do
    echo "Positional parameter $param"
  done
  if $flag
  then
    echo "The option  -m has been used"
  fi
\end{Verbatim}

As seen in line \ref{getopt:use}
\begin{description}
  \item[Line \ref{getopt:use}] uses
\end{description}

\section{Exercises}
\subsection{Lab03 -- Shell script exercises}
\paragraph{Script to tidy files.}  Copies files from a working directory into subfolders, pairing a subfolder with an extension.
\subparagraph{Synopsis} \texttt{tidy \optional{-d work-dir} \optional{-t \required{dest-dir}\required{.ext} }}

\subparagraph{Example} \texttt{tidy -d project -t src.c -t src.h -t out.o}\\
copies C source and header files to the \texttt{src} directory, and object (\texttt{.o}) files to the \texttt{out} directory.
\begin{Verbatim}[numbers=left,gobble=2]
  #!/usr/bin/bash
  declare -A tidypart=()  # Associative array key=value, dir=ext
  while getopts d:t: flag
  do
          case $flag in
                  d)
                          workingdir=$OPTARG
                          ;;
                  t)
                          key=${OPTARG%.*}
                          value=${OPTARG#$key}
                          tidypart[$key]=$value
                          ;;
          esac
  done

  for tgt in ${!tidypart[*]}
  do
          echo "move ${tidypart[$tgt]} to $tgt"
  done
\end{Verbatim}
Version 2.  \texttt{getopts} processes each parameter in turn, until it finds a non-option that isn't a required parameter.
\begin{Verbatim}[numbers=left,gobble=2]
  #!/use/bin/bash
  working=.
  while getopts d:t:e:h spec
  do
      case $spec in
          d)
              working=$OPTARG
              ;;
          t)
              target=$OPTARG
              ;;
          e)
              extension=$OPTARG
              mv $working/*.$extension $target
              ;;
          h)
              echo<<-help
              Usage:  tidy [-d working] [-t tgt] [-e ext] -h
              Options are:
                  -d sets the working directory
                  -t sets the target directory (until next occurrence of -t)
                  -e extensions of files to copy from working to target (no dot needed)
                  -h this help text
              help
  done
\end{Verbatim}

\section{Disk Tidy}
\paragraph{Tidy Files from a directory} into subdirectories based on filename extension.  For example:
\begin{quote}
  Copy all \texttt{.c} files into the \texttt{src} subdirectory and all \texttt{.o} files into the \texttt{obj} subdirectory
\end{quote}
\newcommand{\command}[1]{\texttt{#1}}
\newcommand{\optionarg}[2]{$[$\texttt{#1} \texttt{\emph{#2}}$]$}
\newcommand{\argument}[1]{$\langle$\texttt{\emph{#1}}$\rangle$}
\newcommand{\arguments}[1]{\argument{#1}\ldots}

\paragraph{Synopsis}\command{tidy} \optionarg{-d}{dir} \optionarg{-t}{tgt} \arguments{ext}
\paragraph{Options}\begin{itemize}
  \item[\texttt{-d}] the directory to copy from \texttt{\emph{dir}} is the directory to copy from
  \item[\texttt{-t}] target directory to copy files into \texttt{\emph{tgt}} is the directory
  \item[\texttt{\emph{ext}}] is a list of extensions to copy, files match the glob \texttt{*ext} (\emph{ie.} the extesion should include any dot)
\end{itemize}

\begin{tcolorbox}
  \verbatiminput{dtidy}
\end{tcolorbox}


\end{document}
