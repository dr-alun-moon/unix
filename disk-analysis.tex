\documentclass[12pt]{article}
\usepackage[british]{babel}

\title{Disk Analysis Excercise/Assignment \\---\\ Some updates}
\author{Dr Alun Moon}

\usepackage{tcolorbox,xcolor}

\begin{document}
\maketitle

\begin{abstract}
	These are some notes on suggestions to update the solution to the 
	``Disk Analysis'' excercises and assignment
\end{abstract}

Some of the issues I'm aiming to tidy up include
\begin{itemize}
	\item Remove interactive elements, no use of \verb:read -p: etc
	\item Proper handling of commandline options, make use of \verb:getopts:
	\item Use a \emph{sub-command} structure, like \verb:git:
	\item Allow for some automated testing and marking of basic functionallity
\end{itemize}

\section{Global Option processing}
Options are processed using the \verb:getopts: command.  The recommended pattern is 
\begin{tcolorbox}[title=\texttt{getopts} handling]
\begin{verbatim}
while getopts vf:d: flags
do
    case $flags in
        v)
            VERBOSE=-v
            ;;
        f)
            FILE=$OPTARG
            ;;
        d)
            MOUNTDIR=$OPTARG
            ;;
    esac
done
let options=OPTIND-1
shift $options
\end{verbatim}
\end{tcolorbox}
The use of \verb:let: and \verb:shift: calculates the number of arguments processed 
(\verb:OPTIND: is the index of the next argument to process)
and removes the processed parameters from the parameter list.
The use of \verb:let: allows for arthmetic expressions to be used to set parameters.
The remaining positional parameters can be processed with \verb:\$*:

\section{Sub commands}
The pattern of using sub-commands (\textit{c.f.} \verb:git:) can be implemented quite simply
each sub-command is a shell function defined in the script.

If the first positional parameter is the sub-command, and the remaining positional parameters are the 
options and parameters for the sub-command.  Then it can be evaluated using 
\begin{tcolorbox}[title=sub-command evaluation]
\begin{verbatim}
eval "$@"
\end{verbatim}
\end{tcolorbox}
Using the pattern \verb:"$@": ensures than any parameters with spaces in are properly quoted 
and not subject to the shell's word splitting mechanism.
\subsection{Sub-Command Options}
Options can be handled in functions, in the same way as shell parameters.
Two things to note are:
\begin{itemize}
\item \verb:getopts: works on the current set of positional parameters (\textit{i.e} the function parameters)
\item \verb:OPTIND: is global to the shell so will need resetting prior to invocation of \verb:getopts:
\end{itemize}
\begin{tcolorbox}[title=function option handling]
\begin{verbatim}
function init {
    local OPTIND=1
    while getopts i:v flags
    # etc
}
\end{verbatim}
\end{tcolorbox}
Making a \verb:local: version of \verb:OPTIND: resets the behaviour of \verb:getopts: to process the
current (local) positional parameters.

\begin{tcolorbox}[colframe=red]
This mechanism for command dispatch is rather crude, there isn't any checking for a errors
beyond the shell complaining if the command (function) is not found.

Would a switch be better?
\begin{tcolorbox}
\begin{verbatim}
# options processed
case $1 in
    init)
         init  # call init function
         ;;
    mount)
        shift 1
        mount "$@"
        ;;
    *)
        echo "some error condition, "
        echo "command not recognised?"
        ;;
esac
\end{verbatim}
\end{tcolorbox}

\end{tcolorbox}

\section{Actions}
Looking through the tasks for the seminar session, the assessed practical, and the assignment brief.
There are the following tasks that the student and script must perform.
\begin{enumerate}
	\item Install the script
		\begin{itemize}
			\item copy it to the right directory
			\item set the permissions (execute bit)
			\item add the directory to the \verb:PATH:
		\end{itemize}
		This can't easially be made a function of the script,
	\item Create directories for the lab session.\\
		Make a directory \texttt{/home/student/id/kf4005}
	\item Obtain the raw disk image, put it into the working directory
	\item Examine the partition table
	\item Mount the disk image
	\item Show the mount table
	\\\hrule
	\item Disk analysis script
	\begin{enumerate}
		\item mount the image on a directory --- take precautions
		\item extract metedata from cli specified directories on the image
		\item save metadata to file
		\item Create an SQL database
		\item Create a data table
		\item populate the table from metadata
		\item use SQL SELECT to query table and write to file
	\end{enumerate}
	\item create tar file for submission.	
\end{enumerate}
\end{document}
